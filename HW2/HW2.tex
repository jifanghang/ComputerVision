\documentclass[a4paper]{article}
\usepackage[pdftex]{hyperref}
\usepackage[utf8]{inputenc}
\usepackage[english]{babel}
\usepackage{a4wide}
\usepackage{amsmath}
\usepackage{amssymb}
\usepackage{algorithmic}
\usepackage{algorithm}
\usepackage{ifthen}
% move the asterisk at the right position
%\lstset{language=C,basicstyle=\ttfamily}
\usepackage{moreverb}
\usepackage{palatino}
\usepackage{multicol}
\usepackage{tabularx}
\usepackage{comment}
\usepackage{verbatim}
\usepackage{color}
\usepackage{enumerate}
\usepackage{float}
\usepackage{hyperref}
\usepackage[utf8]{inputenc}
\usepackage{minted}


%% pdflatex?
\newif\ifpdf
\ifx\pdfoutput\undefined
\pdffalse % we are not running PDFLaTeX
\else
\pdfoutput=1 % we are running PDFLaTeX
\pdftrue
\fi
\ifpdf
\usepackage[pdftex]{graphicx}
\else
\usepackage{graphicx}
\fi
\ifpdf
\DeclareGraphicsExtensions{.pdf, .jpg}
\else
\DeclareGraphicsExtensions{.eps, .jpg}
\fi

\parindent=0cm
\parskip=0cm

\setlength{\columnseprule}{0.4pt}
\addtolength{\columnsep}{2pt}

\addtolength{\textheight}{5.5cm}
\addtolength{\topmargin}{-26mm}
\pagestyle{empty}

%%
%% Sheet setup
%% 
\newcommand{\coursename}{Computer Vision}
\newcommand{\courseno}{CO22-320671}

\newcommand{\sheettitle}{Homework}
\newcommand{\mytitle}{}
\newcommand{\mytoday}{\textcolor{blue}{Nov, 8}, 2017}

% Current Assignment number
\newcounter{assignmentno}
\setcounter{assignmentno}{2}

% Current Problem number, should always start at 1
\newcounter{problemno}
\setcounter{problemno}{1}

%%
%% problem and bonus environment
%%
\newcounter{probcalc}
\newcommand{\problem}[2]{
	\pagebreak[2]
	\setcounter{probcalc}{#2}
	~\\
	{\large \textbf{Problem \textcolor{blue}{\arabic{assignmentno}}.\textcolor{blue}{\arabic{problemno}}} \hspace{0.2cm}\textit{#1}} \refstepcounter{problemno}\vspace{2pt}\\}

\newcommand{\bonus}[2]{
	\pagebreak[2]
	\setcounter{probcalc}{#2}
	~\\
	{\large \textbf{Bonus Problem \textcolor{blue}{\arabic{assignmentno}}.\textcolor{blue}{\arabic{problemno}}} \hspace{0.2cm}\textit{#1}} \refstepcounter{problemno}\vspace{2pt}\\}

%% some counters  
\newcommand{\assignment}{\arabic{assignmentno}}

%% solution  
\newcommand{\solution}{\pagebreak[2]{\bf Solution:}\\}

%% Hyperref Setup
\hypersetup{pdftitle={Homework \assignment},
	pdfsubject={\coursename},
	pdfauthor={},
	pdfcreator={},
	pdfkeywords={Computer Architecture and Programming Languages},
	%  pdfpagemode={FullScreen},
	%colorlinks=true,
	%bookmarks=true,
	%hyperindex=true,
	bookmarksopen=false,
	bookmarksnumbered=true,
	breaklinks=true,
	%urlcolor=darkblue
	urlbordercolor={0 0 0.7}
}

\begin{document}
	
	
	
	\coursename \hfill Course: \courseno\\
	Jacobs University Bremen \hfill \mytoday\\
	\textcolor{blue}{Fanlin Wang \& Fanghang Ji}\hfill
	\vspace*{0.3cm}\\
	\begin{center}
		{\Large \sheettitle{} \textcolor{blue}{\assignment}\\}
	\end{center}
	
	\problem{}{0}
	The \textsl{predict} function is implemented as follows in opencv:
\begin{minted}{c++}
void Fisherfaces::predict(InputArray _src, Ptr<PredictCollector> collector) const {
Mat src = _src.getMat();
// check data alignment just for clearer exception messages
if(_projections.empty()) {
// throw error if no data (or simply return -1?)
String error_message = "This Fisherfaces model is not computed yet. Did you call Fisherfaces::train?";
CV_Error(Error::StsBadArg, error_message);
} else if(src.total() != (size_t) _eigenvectors.rows) {
String error_message = format("Wrong input image size. Reason: Training and Test images must be of equal size! Expected an image with %d elements, but got %d.", _eigenvectors.rows, src.total());
CV_Error(Error::StsBadArg, error_message);
}
// project into LDA subspace
Mat q = LDA::subspaceProject(_eigenvectors, _mean, src.reshape(1,1));
// find 1-nearest neighbor
collector->init((int)_projections.size());
for (size_t sampleIdx = 0; sampleIdx < _projections.size(); sampleIdx++) {
	double dist = norm(_projections[sampleIdx], q, NORM_L2);
	int label = _labels.at<int>((int)sampleIdx);
	if (!collector->collect(label, dist))return;
}
}
\end{minted}
	It has two parameters: 
	\begin{enumerate}
		\item src: Sample image to get a prediction from.
		\item collector: User-defined collector object that accepts all results
	\end{enumerate}
	\textsl{subpaceProject} function projects all training image(stored in eigenvectors) to the lower dimensional data space. Then for i from 0 to c, where c is the size of the row, it computes \textsl{dist} as follows: 
\begin{minted}{c++}
double dist = norm(_projections[sampleIdx], q, NORM_L2);
\end{minted}
	In opencv, a PCA is applied in front of the LDA in order to reduce the feature vectors to about the size of the image-count. Therefore, the \textsl{projections} matrix stores the projections of the original data as PCA.eigenvectors * LDA.eigenvectors: 
\begin{minted}{c++}
gemm(pca.eigenvectors, lda.eigenvectors(), 1.0, Mat(), 0.0, _eigenvectors, GEMM_1_T);
for(int sampleIdx = 0; sampleIdx < data.rows; sampleIdx++) {
	Mat p = LDA::subspaceProject(_eigenvectors, _mean, data.row(sampleIdx));
	_projections.push_back(p);
}
\end{minted} 
	\textsl{q} stores the projections of LDA subspace. The heuristic function used to determine the class of the image is therefore the shortest distance obtained by comparing these two projections. \\\\
	Now written this in equations: \\
	Let A be the matrix composed of normalized eigenvectors of
	\[\Sigma_W^{-1}\Sigma_By = \lambda y\]
	where $\Sigma_W$ is the single value decomposition of scatter matrix within class, and  $\Sigma_B$ is that of scatter matrix between class. $c$ is the number of classes, and we have $c-1$ eigenvectors for this equation. Therefore, 
	\[A = \begin{bmatrix}
	\hat{\alpha_1} &... & \hat{\alpha_{c-1}}
	\end{bmatrix}\]
    \newpage 
	let $\mu$ be an $mn$ by 1 matrix extracted from the grey image: \[\tilde{\mu} = \mu - \bar{mu} = U_{PCA} \begin{bmatrix}
	\alpha_1 \\ \vdots \\ \alpha_{l-c}
	\end{bmatrix}\]
	project this to an $l-c$ dim space we get: \[U_{PCA}^T\tilde{\mu}\] We further project it into a lower $c-1$ dim space: 	
	\[\beta_{ij} = A^{T}U_{PCA}^T(\mu_{ij} - \bar{\mu})\]
	\[\beta_{t} = A^{T}U_{PCA}^T(t - \bar{\mu})\]
$\beta_{t}$ is the projection of an PCA $\cdot$ LDA subspace
	\[\gamma_{t} = A^T(t - \bar{\mu})\]
	$\gamma{t}$ is the projection of LDA subspace. For testing image we predict class as 
	\[\underset{t = 1,...,c}{argmin}||\gamma_{t} - \beta_{t}||^2\]
	
	
\end{document}